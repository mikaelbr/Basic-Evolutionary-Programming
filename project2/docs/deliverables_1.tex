\section{Description of the system}
\label{sec:descsys}

This project is based on the previously developed EA from project 1 A and B. Except for some minor changes 
to the genetic functions (see section \hyperref[sec:geneticchanges]{Minor changes}), the code is identical.

As the system is highly object oriented, all the alterations and implementations specific to any problem,
is extended as sub classes of the main "framework" given by the EA library. 

In this section, an explanation of all the different sub classes can be found, description of the genotype
representation and fitness function. 

\subsection{Extensions to the original code}
\label{sec:extensions}

There are three extending classes that overwrites/extends standard behavior; The individual,
mutation and the plotting. In addition there is a new class for calculating the spike distance, with 
three different methods.

\subsubsection{Indevidual: SpikingNeuron}

This is where most of the magic happens. This class takes care of creating new genotype values (if not set),
and converting from genotype to phenotype. 

The constructor takes the gene size as an argument. This sets the number of bits each gene should have. 
(See \hyperref[sec:genotype]{Genotype representation}). This is used to create a bit string genotype of the size 
$gene size * number of params$. 

In addition to creating the random value the class also have a method for converting from genotype
to phenotype, and a helper method for making each of the parameters fit inside of their given range 
(\autoref{sec:genotype}). 

The to phenotype function handles the calculations regarding the Izhikevich Spiking-Neuron Model. 

\subsubsection{Mutation: SpikeMutation}
The mutation simply injects changes to the genotype by removing one random gene and replacing it
with a new random generated one. 

\subsubsection{Plotter: SpikingNeuronPlotter}
Added methods for plotting out the activation levels. 


\subsubsection{SDMs: Spike Train Distance Metrics}
\label{sec:sdm}
A new class for handling calculations of the spike distance. Methods are 

\begin{enumerate}
	\item Spike Time Distance Metric
	\item Spike Interval Distance Metric
	\item Waveform Distance Metric
\end{enumerate}

These methods are used to calculate the inverted distance between two trains. The result of these
methods are used as fitness. That means that the larger the distance is, the lower the result should be.
Thats why the results are inverted by taking $1/distance$. This way a huge distance will become a very
low fitness value, and vice versa.  \\

In the SDM class, there's also a helper method named $compute\_spike\_times()$. This method
takes a train as the argument, and iterates over the train in a window of size 5. The results
are a generated list of the spike times in the train. 

\subsubsection{Helper functions and the main execution file}

The main execution file handles all the interaction between the command line and the application. As
with the previous project, all values can be changed through flags passed as arguments in the command
line. Full overview of which flags are available, can be found by running the main file, and attaching the $-h$ flag. 

A helper function is created to read the data set. This method takes a number between 1 and 4, as argument and 
returns the data train set according to the argument number. \\

To see available parameters running the application, use the following commands:

\begin{lstlisting}[frame=single,caption={Application usage},language=bash] 
$ chmod +x izhikevich.py
$ ./izhikevich.py -h
\end{lstlisting}

\subsubsection{Minor changes}
\label{sec:geneticchanges}

The crossover functions needed a fix, as they didn't take consideration for a genotype with genes larger
then one bit. So the genes would be splitted up, and that destroyed the evolving. \\

The elitism and truncation methods were also faulty. Elitism now uses the right population group, and the
is no longer part of the parent selection, but rather gets immediately passed as a child. They don't get mutated. 


\subsection{Genotype representation}
\label{sec:genotype}

The genotype is represented as a bit string, as that what I first though of, and read out of the assignment 
diagram, also I have the genetic functions to use on binary values. I'd say it is a real-valued representation. 
Where there are 5 genes representing the parameters used by the model, and each of these genes are 
represented as binary strings.  \\ 

When the model is to run, all the parameters are fitted to match the range for each of the parameters, according
to the ranges defined in the project description. 

\begin{enumerate}

	\item $a  \in  [0.001, 0.2] $
	\item $b  \in  [0.01, 0.3] $
	\item $c  \in  [-80, -30] $
	\item $d  \in  [0.1,10] $
	\item $k  \in  [0.01, 1.0] $

\end{enumerate}
 
 The first gene is parameter a, the second is b, and so on. 
 

\subsection{Fitness function}

As described in \autoref{sec:sdm}, the fitness is calculated by using the distance metric, and inverting it. It needs
to be inverted as the distance metrics returns higher values for worse solutions. \\

The distance metrics, each work in different ways.

\begin{enumerate}

	\item \textbf{Spike Time Distance Metric}: Takes in the two trains, converts both to spike times, sum up the 
	total difference between corresponding spike times. 
	
	\item \textbf{Spike Interval Distance Metric}: Also converts both trains to spike times. With the spike times, 
	it calculates the difference in the length of the time intervals between corresponding spike times. 
	
	\item \textbf{Waveform Distance Metric}: Does not use the spike times. Simply calculates the sum of the 
	differences in the spikes. 
	
\end{enumerate}

