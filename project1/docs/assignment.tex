\section{About the project}
The purpose with this project is to gain expirience with the basic mechanisms underlying EAs
by programming them.

The entire project consists of two parts; A) Implementing basic modular EA , and B) using that 
algorithm to solve a Colonel Blotto problem.  

This document will focus on part A, implementing the basic evolutionary algorithm and testing
that algorithm by solving a One Max problem

\section{EA Overview}
The evolutionary algorithm we're implementing in this assignment includes the following aspects:

\begin{enumerate}

	\item A population for reperesenting indeviduals with genotypes.
	
	\item A development phase, for converting genotypes to phenotypes.
	
	\item A fitness test for calculating fitness of all phenotypes.
	
	\item Selection protocols for finding adults from the indeviduals
	
	\item Selection mechanism for finding parents from the adults. 
	
	\item Reproduction for creating children from parents. Taking parts from each of the parent's genos. 
	
	\item Mutation for altering the childrens genotype. 

\end{enumerate}

\noindent{The EA will loop through these items many generations. There must also exist a plotting 
functionality for analyzing and visualizing the results.}

\section{The One-Max Problem}

For testing out the EA, the One-Max problem will be used. The goal of the One-Max problem is to find a 
bit string consisting only of 1's. The initial genotype will be a random n-bit string. With using the number
of 1's as fitness, throughout the generations the bit strings should evolve to consist only of 1's. 

The programmed EA should be able to solve a 40-bit One-Max problem. 