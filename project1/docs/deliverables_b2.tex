\section{Significance of strategy entropy}

The strategy entropy is defined by the following formula:

\begin{center}
$H(s) = -\sum_{i = 1}^{B} p_{i} \log_{2}(p_{i})$
\end{center}

In the formula, $p_{i}$ is the fraction of the total resource that is devoted to the $i$th battle.
This is the same as the $i$th index of the phenotype. \\

Let's analyze the formula. If the normalized weight is very small or very high, the entropy will get close
to 0. This is due to the $\log_{2}()$. So this means that the strategy isn't a generic "trying to cover it all"
strategy, but rather favors certain battles. This sounds more like a strategy. 

Let's say that the game uses both the troop redistribution and strength reduction. With those two aspects,
the smartest move is to start hard in the first battle and therefor get an advantage in the following battles
with redistribution of the troops and full strength. The entropy says that when it is close to zero, there are
big differences in the $p_i$ therefor more likely to be a good strategy. 
