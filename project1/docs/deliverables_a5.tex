\section{Target bit string as a random vector}
I don't really expect much difference in result. There shouldn't be much different in trying to
match a random bit string. The only problem I can think of is that the count of ones might be
to high, and with no mutation there'll be no turning back. For children of strings with 
mostly 1's, will still have mostly 1's if no mutation is present. \\

I tried the problem with the same parameters as before, to be consistent and have comparable
data. \\

By observing the performance of both target bit string and all ones, they seem very similar for
the set parameters. Both the target bit string (\autoref{fig:onemax-8}) and all ones (\autoref{fig:onemax-9})
performed really well.


\insertplot{onemax-8}{Targeted random bit string for matching instead of all ones.}


\insertplot{onemax-9}{Regular One-Max Problem where fitness is the sum of phenotype vector.}

To try to find difference in performance of the to, I changed up the parameters and had several testruns.
I reduced the population size to 20, and reduced birth probability to 60\%, and set a mutation probability
of 20\%. As it turns out, the one-max problem has an advantage over the random bit string. Often the 
random bit string run would solution, but very inconsistantly. The one-max problem how ever, found a
solution every run. See examples in \autoref{fig:onemax-10} and \autoref{fig:onemax-11}. This might be
an coincidens, and that the one-max runs would have been inconsistent with more runs, but after 20 runs
of each, it seemed pretty consistent. 


\insertplot{onemax-10}{Targeted random bit string. Inconsistent solution. Run with population size 20, 
generations 100, mutation at 20\% and reproduction at 60\%}


\insertplot{onemax-11}{Regular One-Max Problem. Run with population size 20, generations 100, 
mutation at 20\% and reproduction at 60\%}

One would think that with higher mutation rate the random bit string matcher would atleast be as good
as the one-max problem, but it seems as is that's not the case. How ever, with tweaks in the parameter
the bit string solution would also be consistent. So conclusion is that there isn't much increase in 
difficulty of the problem using a random bit string as target.