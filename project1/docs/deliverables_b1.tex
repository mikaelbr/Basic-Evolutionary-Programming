\section{Description of representation}

Since the genome should have B binary genes with a random integer between 0 and 10, we know that 
the total size of the binary string should be $B*4$. This is due to the span of the random values all
can be represented in 4-bit. 

For representing a phenotype we need to find out the original 10-based value for the gene and then
normalize it. So in the $toPhenotype()$ method will loop through the genotype string with an increment
of 4 and convert each gene to a denary value. After the weight is found again we can normalize these
values by dividing all by the sum of the weights. If the sum of the weights equals zero (I.e no forces
in battle), the resources get automatically distributed to all battles. \\

Example: If we have a two battle war, and the random integers selected are 7 and 4, we convert those to 
binary values: 0111 and 0100. These binary values get concatenated to 01110100. This is our genotype.

To convert this to a phenotype we need to convert from binary to denary again and yet again we have 7 
and 4. Now we normalize these values. 7/11 and 4/11: 0.6364 and 0.3636. The vector of these values
([0.6364, 0.3636]) is our phenotype.
