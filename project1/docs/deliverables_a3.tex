\section{Analysis of the performance}

I started, as the assignment described, with selection protocol full generational replacement and the mechanism
fitness-proportionate. With implementation of any reproduction or mutation, the individuals will never change
over the generation. So the fitness-proportionate will find the fittest of the random initial phenotypes, and copy
them over time. Eventually we se that the max fitness is consistent and the average fitness will converge to the 
max fitness. This is clearly observable in \autoref{fig:onemax-1}

\insertplot{onemax-1}{EA with no genetic functions. Population size 200 and 100 generations}

There won't be much sense in trying to increase the population size, without any reproduction or mutation for
each generation. So to implement a reproduction of the simplest form, with one-point crossover, but still no 
mutation. The population size and number of generations remained the same. The slice for the crossover is at
random. The reproduction probability is set at 50\%. 

\insertplot{onemax-2}{Run with pop size 200 and one-pit reproduction prob. at 50\%. Splice at random}

As seen in \autoref{fig:onemax-2}, even at 200 in population size the EA can't find a solution. It seems as the 
point stagnates. After more test runs the EA found the solution some times, but not consistently at all. \\

By adding mutation, with 1 bit randomly inversion, at a probability of 10\%, the results doesn't change much. 
If anything the results get even more inconsistent. That could be intuitiv given that the mutation can make
the genotype both better and worse. 

\insertplot{onemax-3}{Same parameters at \autoref{fig:onemax-2}, but with a mutation probability of 10\%}

When deactivating the mutation, and trying different reproduction functions with different values, I could see
that changing to two-point crossover, still with random slice, didn't change the result or the consistency. But
by changing the birth probability to 100\% I observed that the average max fitness in the later generations,
was generally higher. This is observable in \autoref{fig:onemax-4}

\insertplot{onemax-4}{Two-point crossover, no mutation, birth probability of 100\%}

By observing this, I started playing around with the birth probability and tried different reproduction functions.
When the crossover function was set to uniform, without mutation and a birth probability of 90\% I finally found
a consistent solution. For over 10 runs, the EA found a solution every time. An example run can be viewed in
\autoref{fig:onemax-4}. 

\insertplot{onemax-5}{Uniform crossover, no mutation, birth probability of 90\%}

Now by using these parameters I tried to reduce the population size. As I did, I tried to alter different 
parameters with the different population sizes. Nothing really changed. The lowest population I could get
a consistent solution on was the size of 200. I tried to introduce alteration in the genotypes by mutation
on lower population to compensate for the reduce in potential parents to inherit from and introduce 
potentially better genes, how ever with no real results. \\

\textit{It could seem that there might be a bug in the EA, but after hours of debugging I still couldn't find anything
that might be wrong. The only explanation I might have, is that the fitness-proportionate scaling might have to little
pressure }
